\documentclass[12pt,letterpaper]{article}
\usepackage[utf8]{inputenc}
\usepackage[spanish]{babel}
\usepackage{amsmath}
\usepackage{amsfonts}
\usepackage{amssymb}
\usepackage{makeidx}
\usepackage{array}
\usepackage{multirow}
\usepackage{multicol}
\usepackage{booktabs}
\usepackage{colortbl}
\usepackage{setspace}
\usepackage{graphicx}
\usepackage{float}
\usepackage[xcdraw]{xcolor}
\usepackage{colortbl}
\usepackage[margin=1.5cm]{geometry}
\definecolor{TAREA}{HTML}{FF65D4}
\definecolor{EXAMEN}{HTML}{99CE3E}
\title{
\vspace*{-1.5cm}
\Large \textbf{Universidad Nacional Autónoma de Honduras}\\[0.5cm]
\includegraphics[scale=0.4]{UNAH_Gris}\\[0.1cm]
\Large Escuela de Matemática y Ciencias de la Computación\\
\Large Departamento de Matemática Aplicada\\
\large Matemáticas Discretas, MMM420\\
\large Tercer periodo académico 2024.
}
\begin{document}
\maketitle
\begin{multicols}{2}
\renewcommand{\arraystretch}{1.2}
\begin{center}
\begin{tabular}{c}
\cellcolor{gray!40}{GENERALIDADES DEL CURSO}
\end{tabular}
\end{center}
Esta es la información general del curso.
\begin{itemize}
\item \textbf{Asignatura:} Matemáticas Discretas.
\item \textbf{Créditos:} 4.
\item \textbf{Docente:} Jose Alvarenga.
\item \textbf{Email:} jose.alvarenga@unah.edu.hn
\end{itemize}
\begin{center}
\begin{tabular}{c}
\cellcolor{gray!40}{METODOLOGÍA DE ENSEÑANZA}
\end{tabular}
\end{center}
\hspace*{0.5cm}El proceso de enseñanza-aprendizaje de este espacio de aprendizaje se desarrollará mediante la modalidad presencial. Las actividades se han desarrollado para que el estudiante por medio de diversas prácticas alcance los conocimientos del curso.
\begin{center}
\begin{tabular}{c}
\cellcolor{gray!40}{MATERIALES Y RECURSOS DIDÁCTICOS}
\end{tabular}
\end{center}
\hspace*{0.5cm} Para el proceso de enseñanza se utilizarán los siguientes recursos. 
\begin{itemize}
\item GitHub: Se usará este espacio para subir todos los documentos y materiales de apoyo a la clase.
\item Campus Virtual. En este espacio se asignarán diferentes actividades relacionadas con la clase. 
\end{itemize}
\begin{center}
\begin{tabular}{c}
\cellcolor{gray!40}{CONTENIDOS}
\end{tabular}
\end{center}
\hspace*{0.5cm} Este curso esta diseñado para que el estudiante controle diferentes aspectos técnicos y prácticos referentes a las matemáticas discretas:
\begin{itemize}
\item Técnicas de conteo.
\item Lógica Matemática.
\item Teoría de Conjuntos.
\item Inducción Matemática.
\item Problemas recursivos.
\item Introducción básica a la teoría de números. 
\item Funciones.
\end{itemize} 
%==========================================================================
\begin{center}
\begin{tabular}{c}
\cellcolor{gray!40}{EVALUACIÓN}
\end{tabular}
\end{center}
\vspace*{0.5cm} Para la evalución de los contenidos aprendidos en la clase se considerarán las siguientes actividades:
\begin{table}[H]
\centering
\begin{tabular}{|l|l|c|}
\hline
Actividad  & Horario & Porcentaje\\\hline\hline
Examen I&10:00 a.m.$-$12:00 a.m.&80\%\\
Examen II&10:00 a.m.$-$12:00 a.m.&80\%\\
Examen III&10:00 a.m.$-$12:00 a.m.&80\%\\
Tarea I&05:00 a.m.$-$11:59 p.m.&20\%\\
Tarea II&05:00 a.m.$-$11:59 p.m.&20\%\\
Tarea III&05:00 a.m.$-$11:59 p.m.&20\%\\
Reposición &10:00 a.m.$-$12:00 a.m.&80\%\\\hline
\end{tabular}
\end{table}
%==========================================================================
\hspace*{0.5cm} 
\begin{center}
\begin{tabular}{c}
\cellcolor{gray!40}{PROGRAMACIÓN}
\end{tabular}
\end{center}
\begin{center}
\begin{tabular}{ccccc|ccccc}
\multicolumn{5}{c|}{\cellcolor{red!40}{SEPTIEMBRE}}&\multicolumn{5}{c}{\cellcolor{blue!40}{OCTUBRE}}\\
S.&L&M&M&J&S.&L&M&M&J\\
I&\cellcolor{gray!40}{}&\cellcolor{gray!40}{}&11&12&IV&7&8&9&\cellcolor{TAREA}{10}\\
II&\cellcolor{gray!40}{}&\cellcolor{gray!40}{}&18&\cellcolor{white}{19}&V&14&\cellcolor{gray!40}{}&16&\cellcolor{EXAMEN}{17}\\
III&\cellcolor{white}{23}&24&25&\cellcolor{gray!40}{}&VI&\cellcolor{white}{21}&\cellcolor{white}{22}&\cellcolor{white}{23}&\cellcolor{white}{24}\\
&&&&&VII&\cellcolor{white}{28}&29&30&31\\
\multicolumn{5}{c|}{\cellcolor{cyan!40}{NOVIEMBRE}}&\multicolumn{5}{c}{\cellcolor{green!40}{DICIEMBRE}}\\
S.&L&M&M&J&S.&L&M&M&J\\
VIII&\cellcolor{TAREA}{4}&5&6&7&XII&\cellcolor{EXAMEN}{2}&3&4&5\\
IX&\cellcolor{EXAMEN}{11}&12&13&\cellcolor{white}{14}&XIII&\cellcolor{EXAMEN}{9}&10&11&12\\
X&18&19&20&\cellcolor{white}{21}&XIV&\cellcolor{gray!40}{16}&\cellcolor{gray!40}{17}&\cellcolor{gray!40}{18}&\\
XI&\cellcolor{TAREA}{25}&26&27&28&&&&&\\
\end{tabular} 
\end{center}
\begin{center}
\begin{tabular}{lp{0.41\textwidth}}
\cellcolor{TAREA}{\ }&Entrega de tarea.\\
\cellcolor{EXAMEN}{\ }&Examen presencial. Duración de dos horas en el horario de clases. El lunes 9 es una fecha donde se hará una reposición.
\end{tabular}
\end{center}
\begin{center}
\begin{tabular}{c}
\cellcolor{gray!40}{BIBLIOGRAFÍAS}
\end{tabular}
\end{center}
\begin{itemize}
\item Johnsonbaugh, R. (2005). Matemáticas discretas. Pearson Educación.
\item Grimaldi, R. P. (1998). Matemáticas discretas y combinatoria: una introducción con aplicaciones. Pearson Educación.
\end{itemize}
\end{multicols}
\begin{center}
\end{center}
\end{document}
